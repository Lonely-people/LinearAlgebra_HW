\documentclass[12pt]{ctexart}
\usepackage{graphicx} % Required for inserting images
\usepackage{amsmath, amssymb, bm}
\usepackage{enumitem}
\usepackage{caption}
\usepackage{subcaption}
\usepackage{graphicx}
\usepackage{placeins}
\usepackage{listings}
\usepackage{xcolor}

\lstset{
  basicstyle=\ttfamily\small,
  breaklines=true,
  frame=single,
  numbers=left,
  numberstyle=\tiny,
  showstringspaces=false
}


\title{\textbf{人工智能代数学基础大作业:线性代数在概率统计中的运用}}
\author{112401520 王郑涵   152401213 邱镇海}

\begin{document}

\maketitle

\newpage

\tableofcontents

\newpage

\section{引言}

\subsection{选题}

本报告选取了 Introduction to Linear Algebra 中 Chapter 12 的内容,报告包括三个部分:

\begin{itemize}
    \item 学习笔记
    \item 课后习题
    \item 学习心得
\end{itemize}

\subsection{分工}
\begin{itemize}
    \item 王郑涵:主要负责了学习笔记书写和编程实验实现。
    \item 邱镇海:主要负责了课后习题解答书写和编程实验书写。
\end{itemize}

\section{学习笔记}

\subsection{均值,方差和概率}

\subsubsection{基本定义}

我们从基本的几个统计量出发。

\begin{itemize}
    \item 样本均值:\\
    对于样本 $x_1, \dots, x_N$,样本均值 $m=\dfrac{1}{N} \sum_{i = 1}^{N} x_i$。
    \item 方差和样本方差:\\
    方差描述了数据围绕均值的离散程度。样本方差描述了相对于样本均值的距离,方差描述了相对于期望的平均距离。\\
    方差定义为 $\sigma^2 = \mathbb{E}((x - m)^2) = \sum_{i = 1}^{n} p_i(x_i - m)^2$。
    \\
    样本方差定义为 $\mathbb{S}^2=\dfrac{1}{N-1} \sum_{i=1}^{N} (x_i-m)^2$。$\mathbb{S}^2$ 是 $\sigma^2$ 的无偏估计。
    \item 协方差:$\sigma_{12}=\sum_{i} \sum_{j} p_{ij} (x_i - m_1)(y_j-m_2)$。
    \item 相关系数:$\rho = \dfrac{\sigma_{xy}}{\sigma_{x} \sigma_{y}}$,其中 $\sigma_{xy}$ 为 $x$,$y$ 的协方差,$\sigma_{x}$,$\sigma_{y}$ 分别为 $x$,$y$ 的标准差。
\end{itemize}

对于概率密度函数,我们能求出均值和方差:

\[m = \mathbb{E}[x] = \int xp(x)dx\]
\[\sigma^2 = \mathbb{E}[(x - m)^2] = \int (x - m)^2p(x)dx\]

\subsubsection{Monte Carlo Estimetion Methods}

科学计算必须处理数据中的误差。金融计算必须处理不确定的数值和不确定的预测。我们需要接受输入中的不确定性,并估计输出中的方差。

通常我们并不知道概率分布 $p(x)$,我们能做的就是通过不同的输入 $b$ 得到若干个输出 $x(b)$,然后通过样本均值来估计期望。这就是蒙特卡洛方法最简单的形式,这样的误差大概是 $\frac{1}{\sqrt{N}}$ 级别的,由于每次得到输出的计算成本可能很大,这样的效率是不够好的。

多层蒙特卡洛(Multilevel Monte Carlo)是更加优秀的方法,我们找到另一个函数 $y(b)$ 使得计算它的成本比 $x(b)$ 小得多,并且两者十分接近。那么我们便可以用 $N$ 次关于 $y(b)$ 的计算和 $N^{\ast
} < N$ 次关于 $x(b)$ 的计算来估计 $\mathbb{E}[X]$。
\begin{equation}
\text{2-level Monte Carlo: }
\mathbb{E}[x] \approx \frac{1}{N} \sum_{k=1}^{N} y(b_k) + \frac{1}{N^{\ast}} \sum_{k=1}^{N^{\ast}} [x(b_k) - y(b_k)].
\end{equation}
设 $y(b)$ 计算成本为 $C$,$x(b)$ 计算成本为 $C^{\ast}$,在总成本固定的前提下,我们可以最小化方差,最佳比例为 $\dfrac{N^{\ast}}{N} = \sqrt{\dfrac{C}{C^{\ast}}} \dfrac{\sigma^{\ast}}{\sigma}$。

\subsection{协方差矩阵}

\subsubsection{定义与性质}

定义:已知试验的结果取值向量 $\boldsymbol{X}$ 以及均值 $\boldsymbol{\bar{X}} = \mathbb{E}[\boldsymbol{X}]$,协方差矩阵为
\begin{equation}
\boldsymbol{V}=\mathbb{E}[(\boldsymbol{X} - \boldsymbol{\bar{X}})(\boldsymbol{X} - \boldsymbol{\bar{X}})^{\mathrm{T}}]
\end{equation}

\subsubsection{性质}

\begin{itemize}
    \item 协方差矩阵\textbf{至少}是半正定的(当变量相互独立时,它是正定的,且是对角矩阵)。协方差矩阵同时也是对称矩阵。
    \item 由于协方差矩阵 $V$ 是对称矩阵,我们可以把它分解成 $\boldsymbol{V}=\boldsymbol{Q} \Lambda \boldsymbol{Q} ^{\mathrm{T}}$。这样我们得到了若干个特征值 $\lambda_i \geq 0$ 和若干个正交的单位特征向量 $q_i$,原来的若干个试验相当于分解得到的若干个独立试验的线性组合。
    \item 对原向量 $\boldsymbol{X}$ 做线性变换得到 $\boldsymbol{Z} = \boldsymbol{A} \boldsymbol{X}$,那么 $\boldsymbol{Z}$ 的协方差矩阵为 
    \begin{equation}
        \boldsymbol{V}_{\boldsymbol{Z}} = \boldsymbol{A} \boldsymbol{V}_{\boldsymbol{X}} \boldsymbol{A}^{\mathrm{T}}
    \end{equation}
\end{itemize}

\subsection{加权最小二乘与卡尔曼滤波}

\subsubsection{加权最小二乘}

我们知道对于一个无解的线性方程 $\boldsymbol{A} \boldsymbol{x} = \boldsymbol{b}$,我们选择 $\hat{\boldsymbol{x}}$ 来最小化误差 $||\boldsymbol{b} - \boldsymbol{A} \boldsymbol{x}||^2$,有方程 $\boldsymbol{A} ^{\mathrm{T}} \boldsymbol{A} \hat{\boldsymbol{x}} = \boldsymbol{A} ^{\mathrm{T}} \boldsymbol{b}$。

当 $\boldsymbol{b}$ 的测量误差是\textbf{相互独立的随机变量},均值为 0,方差为 1 且服从正态分布,那么最小二乘法是合适的。当其不独立或方差不同时,需要使用加权最小二乘法。

误差为
\[
    \boldsymbol{E} = (\boldsymbol{b} - \boldsymbol{A} \boldsymbol{x})^{\mathrm{T}} \boldsymbol{V}^{-1} (\boldsymbol{b} - \boldsymbol{A} \boldsymbol{x})
\]

方程变为
\begin{equation}
    \boldsymbol{A} ^{\mathrm{T}} \boldsymbol{V}^{-1} \boldsymbol{A} \hat{\boldsymbol{x}} = \boldsymbol{A} ^{\mathrm{T}} \boldsymbol{V}^{-1} \boldsymbol{b}
\end{equation}

为了衡量整个试验的可靠性,我们同样考虑 $\hat{\boldsymbol{x}}$ 的均方误差,并将其与输入的协方差矩阵建立联系。我们有矩阵 $\boldsymbol{W}$:
\begin{equation}
    \boldsymbol{W} = \mathbb{E}[(\hat{\boldsymbol{x}} - \boldsymbol{x})(\hat{\boldsymbol{x}} - \boldsymbol{x})^{\mathrm{T}}]=(\boldsymbol{A}^{\mathrm{T}} \boldsymbol{V}^{-1} \boldsymbol{A})^{-1}
\end{equation}
假设 $\boldsymbol{b}$ 的协方差矩阵为 $\boldsymbol{V}$,$\hat{\boldsymbol{x}} = \boldsymbol{Lb}$。由 (3) 可知,$\hat{\boldsymbol{x}}$ 的协方差矩阵为 $\boldsymbol{LV} \boldsymbol{L}^{\mathrm{T}}$。由 (4) 可知
\[
    \boldsymbol{LV} \boldsymbol{L}^{\mathrm{T}} = (\boldsymbol{A}^{\mathrm{T}} \boldsymbol{V}^{-1} \boldsymbol{A})^{-1}\boldsymbol{A}^{\mathrm{T}} \boldsymbol{V}^{-1} \; \boldsymbol{V} \; \boldsymbol{V}^{-1} \boldsymbol{A} (\boldsymbol{A}^{\mathrm{T}} \boldsymbol{V}^{-1} \boldsymbol{A})^{-1} = (\boldsymbol{A}^{\mathrm{T}} \boldsymbol{V}^{-1} \boldsymbol{A})^{-1}
\]

\subsubsection{卡尔曼滤波}

考虑一个动态问题:每次新增一个观测方程,如何根据之前的观测方程,状态转移方程和最优估计 $\hat{x}_{k-1}$ 来更新最优估计 $\hat{x}_k$。
\begin{equation*}
    A_0 x_0 = b_0 \quad
    x_1 = F_0 x_0 \quad
    A_1 x_1 = b_1 \quad
    x_2 = F_1 x_1 \quad
    A_2 x_2 = b_2
\end{equation*}
\[
    x_{\text{new}} = F_{\text{old}} x_{\text{old}}
\]
考虑简化的问题:不考虑矩阵 $F$。我们初始有 $A_0x_0 = b_0$,有加权最小二乘方程 
\begin{equation}
    A_0^{\mathrm{T}} V_0^{-1} A_0 \hat{x}_0 = A_0^{\mathrm{T}} V_0^{-1} b_0
\end{equation}
现在新增一个方程 \[
    \begin{bmatrix}
    A_0 \\
    A_1
    \end{bmatrix}
    \hat{x}_1
    =
    \begin{bmatrix}
    b_0 \\
    b_1
    \end{bmatrix}
\]
有加权方程
\begin{equation}
    \begin{bmatrix}
    A_0^{T} & A_1^{T}
    \end{bmatrix}
    \begin{bmatrix}
    V_0^{-1} & \\
    & V_1^{-1}
    \end{bmatrix}
    \begin{bmatrix}
    A_0 \\
    A_1
    \end{bmatrix}
    \hat{x}_1
    =
    \begin{bmatrix}
    A_0^{T} & A_1^{T}
    \end{bmatrix}
    \begin{bmatrix}
    V_0^{-1} & \\
    & V_1^{-1}
    \end{bmatrix}
    \begin{bmatrix}
    b_0 \\
    b_1
    \end{bmatrix}.
\end{equation}
有卡尔曼更新方程
\begin{equation}
    \hat{x}_1 = \hat{x}_0 + K_1(b_1-A_1\hat{x}_0)
\end{equation}
其中 $K_1$ 是卡尔曼增益矩阵,$K_1$ 又可以从方差-协方差矩阵 $W_1$ 的逆得到:
\begin{equation}
    W_1^{-1} = W_0^{-1} + A_1^{\mathrm{T}} V_1^{-1}A_1
\end{equation}
\begin{equation}
    K_1 = W_1 A_1^{\mathrm{T}} V_1^{-1}
\end{equation}

\section{课后习题解答}

\subsection{习题1}

已知样本方差 $S^2$ 有如下两种等价形式(其中 $m$ 为样本均值):
\[
S^2=\frac{1}{N-1}\sum_{i=1}^{N}(x_i-m)^2
=\frac{1}{N-1}\left[\left(\sum_{i=1}^{N}x_i^2\right)-Nm^2\right].
\]

请验证:期望方差(加权方差)$\sigma^2$ 也有对应的等价恒等式。设
\[
m=\sum_i p_i x_i,
\]

证明
\[
\sigma^2=\sum_i p_i(x_i-m)^2=\left(\sum_i p_i x_i^2\right)-m^2.
\]
\textbf{解答:}

从期望方差(加权方差)的定义出发:
\[
\sigma^2=\sum_i p_i(x_i-m)^2.
\]

先展开平方项:
\begin{align*}
\sigma^2
&=\sum_i p_i\bigl(x_i^2-2x_i m+m^2\bigr) \\
&=\sum_i p_i x_i^2-2m\sum_i p_i x_i+m^2\sum_i p_i.
\end{align*}

注意到给定条件
\[
\sum_i p_i x_i=m
\quad\text{且}\quad
\sum_i p_i=1,
\]

代入可得
\begin{align*}
\sigma^2
&=\sum_i p_i x_i^2-2m(m)+m^2(1) \\
&=\sum_i p_i x_i^2-2m^2+m^2 \\
&=\sum_i p_i x_i^2-m^2.
\end{align*}

因此恒等式成立:
\[
\sigma^2=\left(\sum_i p_i x_i^2\right)-m^2.
\]

% \textbf{结论:}(一句话总结)

\subsection{习题2}
\textbf{题目:}\\

掷一枚公平硬币 $N$ 次,得到 $i$ 次正面朝上的概率为
\[
p_i=\frac{b_i}{2^N},\qquad b_i=\binom{N}{i}=\frac{N!}{i!(N-i)!}.
\]

由于
\[
\sum_{i=0}^{N} b_i=(1+1)^N=2^N,
\]

所以
\[
\sum_{i=0}^{N} p_i=1.
\]

并且注意到 $b_i=b_{N-i}$。

\textbf{问题:} 证明该分布的均值
\[
m=0\cdot p_0+1\cdot p_1+\cdots+N\cdot p_N=\sum_{i=0}^{N} i\,p_i
\]

等于 $\dfrac{N}{2}$。

\textbf{解答:}\\

我们计算
\[
m=\sum_{i=0}^{N} i\,p_i=\frac{1}{2^N}\sum_{i=0}^{N} i\binom{N}{i}.
\]

利用组合恒等式
\[
i\binom{N}{i}=N\binom{N-1}{i-1},
\]

于是
\[
\sum_{i=0}^{N} i\binom{N}{i}
=\sum_{i=1}^{N} N\binom{N-1}{i-1}
= N\sum_{j=0}^{N-1}\binom{N-1}{j}
= N\cdot 2^{\,N-1},
\]

其中令 $j=i-1$,并用到了
\[
\sum_{j=0}^{N-1}\binom{N-1}{j}=(1+1)^{N-1}=2^{\,N-1}.
\]

因此
\[
m=\frac{1}{2^N}\cdot N\cdot 2^{\,N-1}=\frac{N}{2}.
\]

所以均值确实为 $\dfrac{N}{2}$。
% \textbf{结论:}

\subsection{习题3}

\textbf{题目:}\\

对任意函数 $f(x)$,其期望为
\[
\mathbb{E}[f]=\sum_i p_i f(x_i)\quad \text{或}\quad \mathbb{E}[f]=\int p(x)f(x)\,dx
\]

(分别对应离散概率与连续概率)。设均值为 $\mathbb{E}[x]=m$,方差为
\[
\mathbb{E}\big[(x-m)^2\big]=\sigma^2.
\]

求 $\mathbb{E}[x^2]$。

\textbf{解答:}\\

由方差定义
\[
\sigma^2=\mathbb{E}\big[(x-m)^2\big]
=\mathbb{E}\big[x^2-2mx+m^2\big]
=\mathbb{E}[x^2]-2m\mathbb{E}[x]+m^2.
\]

又因为 $\mathbb{E}[x]=m$,代入得
\[
\sigma^2=\mathbb{E}[x^2]-2m^2+m^2=\mathbb{E}[x^2]-m^2.
\]

因此
\[
\\mathbb{E}[x^2]=\sigma^2+m^2.
\]

\subsection{习题4}
\textbf{题目:}\\

对于 $M=3$ 个相互独立的实验,其均值分别为 $m_1,m_2,m_3$,方差分别为$\sigma_1^2,\sigma_2^2,\sigma_3^2$。求它们的协方差矩阵 $V$。\\

\textbf{解答:}\\

由于三个实验相互独立,因此不同实验之间协方差为 0:
\[
\mathrm{Cov}(X_i,X_j)=0,\quad i\neq j
\]

每个实验的方差为:
\[
\mathrm{Var}(X_i)=\sigma_i^2
\]

因此协方差矩阵为对角矩阵:
\[
V=
\begin{pmatrix}
\sigma_1^2 & 0 & 0 \\
0 & \sigma_2^2 & 0 \\
0 & 0 & \sigma_3^2
\end{pmatrix}
\]


\subsection{习题5}
\textbf{题目:}\\

对同一个变量 $x$ 进行两次测量,得到两条观测方程
\[
x=b_1,\qquad x=b_2.
\]

假设两次测量误差均值为零,方差分别为 $\sigma_1^2$ 与 $\sigma_2^2$,且两次误差相互独立。

因此误差协方差矩阵
\[
V=\begin{pmatrix}
\sigma_1^2 & 0\\
0 & \sigma_2^2
\end{pmatrix}.
\]

将两条方程写成 $Ax=b$ 的形式(其中 $A$ 是 $2\times 1$ 矩阵)。

参照教材 Example 1,求基于 $b_1,b_2$ 的最优估计 $\hat{x}$,以及估计误差的方差
\[
\mathbb{E}\!\left[\hat{x}\hat{x}^T\right].
\]

\textbf{解答:}\\

\textbf{(1) 写成矩阵形式}

两条观测方程为
\[
x=b_1,\qquad x=b_2.
\]

写成 $Ax=b$ 形式:
\[
A=\begin{pmatrix}1\\1\end{pmatrix},\qquad
b=\begin{pmatrix}b_1\\b_2\end{pmatrix}.
\]

误差协方差矩阵为
\[
V=\begin{pmatrix}
\sigma_1^2 & 0\\
0 & \sigma_2^2
\end{pmatrix},\qquad
V^{-1}=\begin{pmatrix}
\frac{1}{\sigma_1^2} & 0\\
0 & \frac{1}{\sigma_2^2}
\end{pmatrix}.
\]

\textbf{(2) 最小二乘(加权)最优估计}

加权最小二乘解为
\[
\hat{x} = (A^T V^{-1} A)^{-1} A^T V^{-1} b.
\]

先计算:
\[
A^T V^{-1} A
=
\begin{pmatrix}1&1\end{pmatrix}
\begin{pmatrix}
\frac{1}{\sigma_1^2} & 0\\
0 & \frac{1}{\sigma_2^2}
\end{pmatrix}
\begin{pmatrix}1\\1\end{pmatrix}
=
\frac{1}{\sigma_1^2}+\frac{1}{\sigma_2^2}.
\]

再计算:
\[
A^T V^{-1} b
=
\begin{pmatrix}1&1\end{pmatrix}
\begin{pmatrix}
\frac{1}{\sigma_1^2} & 0\\
0 & \frac{1}{\sigma_2^2}
\end{pmatrix}
\begin{pmatrix}b_1\\b_2\end{pmatrix}
=
\frac{b_1}{\sigma_1^2}+\frac{b_2}{\sigma_2^2}.
\]

因此
\[
\
\hat{x}
=
\frac{\frac{b_1}{\sigma_1^2}+\frac{b_2}{\sigma_2^2}}
{\frac{1}{\sigma_1^2}+\frac{1}{\sigma_2^2}}
.
\]

\textbf{(3) 估计误差方差}

加权最小二乘估计的协方差为
\[
\mathrm{Var}(\hat{x})
=
(A^T V^{-1} A)^{-1}
=
\left(\frac{1}{\sigma_1^2}+\frac{1}{\sigma_2^2}\right)^{-1}.
\]

因此题目所求为
\[
\
\mathbb{E}\!\left[\hat{x}\hat{x}^T\right]
=
\left(\frac{1}{\sigma_1^2}+\frac{1}{\sigma_2^2}\right)^{-1}
.
\]

\subsection{习题6}
\textbf{题目:}\\

若随机变量 $x$ 与 $y$ 相互独立,其边缘概率密度分别为 $p_1(x)$ 与 $p_2(y)$,则联合概率密度
\[
p(x,y)=p_1(x)p_2(y).
\]

通过将二维积分(积分区间均为 $(-\infty,\infty)$)拆分为两个一维积分的乘积,证明:
\[
\iint_{-\infty}^{\infty} p(x,y)\,dx\,dy=1,
\qquad
\iint_{-\infty}^{\infty} (x+y)\,p(x,y)\,dx\,dy=m_1+m_2.
\]

其中
\[
m_1=\int_{-\infty}^{\infty} x\,p_1(x)\,dx,\qquad
m_2=\int_{-\infty}^{\infty} y\,p_2(y)\,dy.
\]

\textbf{解答:}\\

\textbf{(1) 证明归一化:} 由于 $p(x,y)=p_1(x)p_2(y)$,
\[
\iint p(x,y)\,dx\,dy
=
\iint p_1(x)p_2(y)\,dx\,dy.
\]

先对 $x$ 积分、再对 $y$ 积分(或反之),可将其分离为乘积:
\[
\iint p_1(x)p_2(y)\,dx\,dy
=
\left(\int_{-\infty}^{\infty} p_1(x)\,dx\right)
\left(\int_{-\infty}^{\infty} p_2(y)\,dy\right).
\]

而 $p_1, p_2$ 都是概率密度,因此各自积分为 1:
\[
\int_{-\infty}^{\infty} p_1(x)\,dx=1,\qquad
\int_{-\infty}^{\infty} p_2(y)\,dy=1.
\]

故
\[
\iint p(x,y)\,dx\,dy = 1\cdot 1=1.
\]

\textbf{(2) 证明期望可加:}
\[
\iint (x+y)p(x,y)\,dx\,dy
=
\iint (x+y)\,p_1(x)p_2(y)\,dx\,dy.
\]

利用线性性拆分:
\[
\iint (x+y)\,p_1(x)p_2(y)\,dx\,dy
=
\iint x\,p_1(x)p_2(y)\,dx\,dy
+
\iint y\,p_1(x)p_2(y)\,dx\,dy.
\]

对第一项分离变量:
\[
\iint x\,p_1(x)p_2(y)\,dx\,dy
=
\left(\int_{-\infty}^{\infty} x\,p_1(x)\,dx\right)
\left(\int_{-\infty}^{\infty} p_2(y)\,dy\right)
= m_1\cdot 1 = m_1.
\]

对第二项同理:
\[
\iint y\,p_1(x)p_2(y)\,dx\,dy
=
\left(\int_{-\infty}^{\infty} p_1(x)\,dx\right)
\left(\int_{-\infty}^{\infty} y\,p_2(y)\,dy\right)
= 1\cdot m_2 = m_2.
\]

相加得
\[
\iint (x+y)p(x,y)\,dx\,dy = m_1+m_2.
\]

证毕。

\subsection{习题7}
\textbf{题目:}\\

继续第 6 题。设随机变量 $x,y$ 相互独立,因此联合概率密度
\[
p(x,y)=p_1(x)\,p_2(y).
\]

证明
\[
\iint (x-m_1)^2\,p(x,y)\,dx\,dy=\sigma_1^2,\qquad
\iint (x-m_1)(y-m_2)\,p(x,y)\,dx\,dy=0.
\]

因此 $2\times 2$ 的协方差矩阵 $V$ 为对角矩阵,其元素为 \underline{\qquad\qquad}。

\textbf{解答:}\\

已知独立性给出 $p(x,y)=p_1(x)p_2(y)$,并记
\[
m_1=\mathbb E[X],\quad m_2=\mathbb E[Y],\quad 
\sigma_1^2=\mathrm{Var}(X),\quad \sigma_2^2=\mathrm{Var}(Y).
\]

\textbf{(1) 证明 $\iint (x-m_1)^2 p(x,y)\,dx\,dy=\sigma_1^2$:}

\begin{align*}
\iint (x-m_1)^2 p(x,y)\,dx\,dy
&=\iint (x-m_1)^2 p_1(x)p_2(y)\,dx\,dy \\
&=\int (x-m_1)^2 p_1(x)\,dx \;\; \int p_2(y)\,dy \\
&=\int (x-m_1)^2 p_1(x)\,dx \\
&=\mathrm{Var}(X)=\sigma_1^2,
\end{align*}

其中用到 $\int p_2(y)\,dy=1$。

\textbf{(2) 证明 $\iint (x-m_1)(y-m_2) p(x,y)\,dx\,dy=0$:}

\begin{align*}
\iint (x-m_1)(y-m_2) p(x,y)\,dx\,dy
&=\iint (x-m_1)(y-m_2) p_1(x)p_2(y)\,dx\,dy \\
&=\left(\int (x-m_1)p_1(x)\,dx\right)
  \left(\int (y-m_2)p_2(y)\,dy\right) \\
&=\bigl(\mathbb E[X-m_1]\bigr)\bigl(\mathbb E[Y-m_2]\bigr) \\
&=(\mathbb E[X]-m_1)(\mathbb E[Y]-m_2)=0.
\end{align*}

因此协方差矩阵
\[
V=
\begin{pmatrix}
\mathrm{Var}(X) & \mathrm{Cov}(X,Y)\\
\mathrm{Cov}(Y,X) & \mathrm{Var}(Y)
\end{pmatrix}
=
\begin{pmatrix}
\sigma_1^2 & 0\\
0 & \sigma_2^2
\end{pmatrix}.
\]

\subsection{习题8}
\textbf{题目:}\\

设二维协方差矩阵
\[
V=\begin{pmatrix}
\sigma_1^2 & \sigma_{12}\\
\sigma_{12} & \sigma_2^2
\end{pmatrix},
\]

证明其逆矩阵可写为
\[
V^{-1}
=\begin{pmatrix}
\sigma_1^2 & \sigma_{12}\\
\sigma_{12} & \sigma_2^2
\end{pmatrix}^{-1}
=\frac{1}{1-\rho^2}
\begin{pmatrix}
\frac{1}{\sigma_1^2} & -\frac{\rho}{\sigma_1\sigma_2}\\[6pt]
-\frac{\rho}{\sigma_1\sigma_2} & \frac{1}{\sigma_2^2}
\end{pmatrix},
\quad
\rho=\frac{\sigma_{12}}{\sigma_1\sigma_2}.
\]

并指出该形式会出现在二维高斯分布指数项
\[
-(x-m)^\top V^{-1}(x-m)
\]

中。

\textbf{解答:}\\

对任意 $2\times 2$ 矩阵
\[
\begin{pmatrix}
a & b\\
b & d
\end{pmatrix},
\]

若行列式 $ad-b^2\neq 0$,则其逆为
\[
\begin{pmatrix}
a & b\\
b & d
\end{pmatrix}^{-1}
=\frac{1}{ad-b^2}
\begin{pmatrix}
d & -b\\
-b & a
\end{pmatrix}.
\]

令 $a=\sigma_1^2,\; b=\sigma_{12},\; d=\sigma_2^2$,得到
\[
V^{-1}
=\frac{1}{\sigma_1^2\sigma_2^2-\sigma_{12}^2}
\begin{pmatrix}
\sigma_2^2 & -\sigma_{12}\\
-\sigma_{12} & \sigma_1^2
\end{pmatrix}.
\]

又由相关系数定义 $\rho=\dfrac{\sigma_{12}}{\sigma_1\sigma_2}$,可写
\[
\sigma_{12}=\rho\,\sigma_1\sigma_2,
\qquad
\sigma_1^2\sigma_2^2-\sigma_{12}^2
=\sigma_1^2\sigma_2^2\bigl(1-\rho^2\bigr).
\]

代回并将分母因子拆开:
\[
V^{-1}
=\frac{1}{\sigma_1^2\sigma_2^2(1-\rho^2)}
\begin{pmatrix}
\sigma_2^2 & -\rho\sigma_1\sigma_2\\
-\rho\sigma_1\sigma_2 & \sigma_1^2
\end{pmatrix}.
\]

将 $\sigma_1^2\sigma_2^2$ 分配进矩阵元素(逐项相除):
\[
V^{-1}
=\frac{1}{1-\rho^2}
\begin{pmatrix}
\frac{\sigma_2^2}{\sigma_1^2\sigma_2^2} & -\frac{\rho\sigma_1\sigma_2}{\sigma_1^2\sigma_2^2}\\[6pt]
-\frac{\rho\sigma_1\sigma_2}{\sigma_1^2\sigma_2^2} & \frac{\sigma_1^2}{\sigma_1^2\sigma_2^2}
\end{pmatrix}
=\frac{1}{1-\rho^2}
\begin{pmatrix}
\frac{1}{\sigma_1^2} & -\frac{\rho}{\sigma_1\sigma_2}\\[6pt]
-\frac{\rho}{\sigma_1\sigma_2} & \frac{1}{\sigma_2^2}
\end{pmatrix}.
\]

证毕。

\section{编程实验}

\subsection{实验:加权最小二乘}

\subsubsection{实验目的}
本实验为加权最小二乘法和普通最小二乘法的对比:
\begin{enumerate}
    \item 构造带异方差噪声的线性回归模型,噪声方差随样本变化;
    \item 对比普通最小二乘与加权最小二乘的参数估计差异;
\end{enumerate}

\subsubsection{理论模型}

假设带异方差噪声的一元线性模型:
\begin{equation*}
y = ax + b + \varepsilon, 
\qquad \mathbb{E}[\varepsilon_i]=0,\qquad \mathrm{Var}(\varepsilon_i)=\sigma_i^2.
\end{equation*}

将其写成矩阵形式。令
\[
\bm{y}=
\begin{bmatrix}
y_1\\ \vdots\\ y_N
\end{bmatrix},\quad
A=
\begin{bmatrix}
x_1 & 1\\
\vdots & \vdots\\
x_N & 1
\end{bmatrix},\quad
\bm{\beta}=
\begin{bmatrix}
a\\ b
\end{bmatrix},\quad
\bm{\varepsilon}=
\begin{bmatrix}
\varepsilon_1\\ \vdots\\ \varepsilon_N
\end{bmatrix},
\]

则
\begin{equation*}
\bm{y} = A\bm{\beta} + \bm{\varepsilon}.
\end{equation*}

\textbf{普通最小二乘:}
通过最小化平方残差 $\|\bm{y}-A\bm{\beta}\|_2^2$ 得到解
\begin{equation*}
\hat{\bm{\beta}}_1=(A^{\mathrm T}A)^{-1}A^{\mathrm T}\bm{y}.
\end{equation*}

\textbf{加权最小二乘:}
若噪声协方差矩阵为
\[
V=\mathrm{diag}(\sigma_1^2,\ldots,\sigma_N^2),
\]

则权重矩阵取
\[
W = V^{-1}=\mathrm{diag}\!\left(\frac{1}{\sigma_1^2},\ldots,\frac{1}{\sigma_N^2}\right),
\]

最小化加权残差二次型 $(\bm{y}-A\bm{\beta})^{\mathrm T}W(\bm{y}-A\bm{\beta})$,其解为
\begin{equation*}
\hat{\bm{\beta}}_2=(A^{\mathrm T}WA)^{-1}A^{\mathrm T}W\bm{y}.
\end{equation*}

\subsubsection{数据构造}
\begin{enumerate}
    \item 生成自变量 $x_i$:在区间 $[x_{\min},x_{\max}]$ 内均匀采样;
    \item 设定真实参数 $a,b$,构造真实直线 $y_i^\star=ax_i+b$;
    \item 令噪声标准差随 $x$ 变化:
    \begin{equation*}
        \sigma(x)=\sigma_0\bigl(1+k|x|\bigr),
    \end{equation*}
    从而 $\sigma_i=\sigma(x_i)$;
    \item 对每个样本生成噪声 $\varepsilon_i\sim \mathcal{N}(0,\sigma_i^2)$,并得到观测值
    \begin{equation*}
        y_i = y_i^\star + \varepsilon_i = ax_i+b+\varepsilon_i.
    \end{equation*}
    \item 参数:
        \begin{itemize}
            \item $N = 80$,$a = 2.0$,$b = -1.0$;
            \item $x_{\min} = -3.0$,$x_{\max} = 3.0$;
            \item $\sigma_0 = 0.06$,$k = 20$。
        \end{itemize}
\end{enumerate}

\subsubsection{实验步骤}
\begin{enumerate}
    \item 生成带异方差噪声的数据集 $\{(x_i,y_i,\sigma_i)\}_{i=1}^N$;
    \item 分别用两种方法估计参数,得到
    \[
    \hat{\bm{\beta}}_1=
    \begin{bmatrix}\hat a_1\\ \hat b_1\end{bmatrix},
    \qquad
    \hat{\bm{\beta}}_2=
    \begin{bmatrix}\hat a_2\\ \hat b_2\end{bmatrix};
    \]
    \item 计算并对比参数误差:
    \[
    |\hat a_1-a|,\;|\hat b_1-b|,\qquad
    |\hat a_2-a|,\;|\hat b_2-b|;
    \]
    \item 计算预测误差(相对于无噪声真值直线 $y^\star=ax+b$)的均方误差:
    \begin{equation}
        \mathrm{MSE}(\hat{\bm{\beta}})=\frac{1}{N}\sum_{i=1}^{N}\bigl(\hat a x_i+\hat b-(ax_i+b)\bigr)^2.
    \end{equation}
\end{enumerate}

\subsubsection{可视化结果与解释}

图 \ref{fig:scatter} 为异方差数据散点图。可以观察到,当 $|x|$ 较大时,样本点围绕真实直线的离散程度明显增大;而在 $|x|$ 较小的区域,数据点更加集中,对应较小的噪声方差。

图 \ref{fig:weights} 展示了权重 $w_i=1/\sigma_i^2$ 随 $x_i$ 的变化关系。可以看到,在噪声方差较小的区域,权重更大;在噪声方差较大的区域,权重显著减小。这表明加权最小二乘会在拟合过程中更信任噪声较小的样本点,从而降低高噪声样本对参数估计的干扰。

\begin{figure}[htp]
    \centering
    \centering
    \includegraphics[width=\textwidth]{Figure_1.png}
    \caption{数据分布}
    \label{fig:scatter}
\end{figure}

\begin{figure}[htp]
    \centering
    \centering
    \includegraphics[width=\textwidth]{Figure_2.png}
    \caption{权重曲线}
    \label{fig:weights}
\end{figure}

\FloatBarrier

\paragraph{代码实现}
实验中数据生成与可视化的完整实现见 Listing \ref{lst:wls_codefile}。

\lstinputlisting[
    language=Python,
    caption={main.py},
    label={lst:wls_codefile}
]{main.py}

\section{学习心得}

本次学习感受最深的是协方差矩阵的分解,它把概率统计中的“独立”和线性代数中的正交对角化关联起来,是个非常有用的方法。

其次蒙特卡洛方法让我初步认识到了在科学计算中,可以从计算结构上改进,用便宜的近似模型提供主要贡献,再用少量昂贵模型修正偏差。

最后就是加权最小二乘法,让我学到了通过噪声方差的大小来加权估计,方差越小的数据越可靠,权重应该越大,这样可以得到更好的估计。

\end{document}